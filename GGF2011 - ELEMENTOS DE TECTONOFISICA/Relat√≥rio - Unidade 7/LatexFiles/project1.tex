%\documentclass[paper,twocolumn]{geophysics}
\documentclass[manuscript]{geophysics}[10pt]
%\documentclass[long]{geophysics}
% An example of defining macros

\usepackage[portuguese]{babel}
\usepackage[T1]{fontenc}
\usepackage[utf8]{inputenc}

\usepackage{amsmath}

\newcommand{\rs}[1]{\mathstrut\mbox{\scriptsize\rm #1}}
\newcommand{\rr}[1]{\mbox{\rm #1}}
\usepackage{color}

\begin{document}

\begin{center}
\textbf{\LARGE
 Artigo 07 - Earthquake stress drops, ambient tectonic stresses and stresses that drive plate motions - \cite{hanks_earthquake_1977}} \\
\textit{Diogo Luiz de Oliveira Coelho}
\footnote{\textit{Universidade Federal do Rio Grande do Norte - UFRN
Centro de Ciências Exatas e da Terra - CCET
Departamento de Geofísica
Campus Universitário - Lagoa Nova
59072-970 Natal, RN}}
\end{center} 



O artigo apresentado não conseguiu expor claramente suas ideas. A leitura do texto foi confusa e cansativa, muitas vezes, porque o autor não foi conciso na escrita. A necessidade de uma releitura de cada parágrafo era constante devido a grandes sentenças e excesso de informações. Isso é marcado quando se lê o título e as conclusões e não se consegue ver com clareza o que foi feito. Ratificando isso, a leitura do resumo também não é clara, a utilização da palavra "se"(\textit{if}) em várias partes do texto gerou uma dificuldade na hora de entender as ideias propostas. É interessante frisar que até mesmo nas ilustrações apresentadas encontra-se uma dificuldade de comunicação. As figuras tem legendas muito grandes, e isso faz com que o artigo tenho uma falta de direcionamento. Um exemplo para essa dificuldade no entendimento é a figura 4, porque entender que é um desenho esquemático das forças que atuam numa placa de comportamento elástico é complicado. A identificação de cada força atuante é prejudicada pelo desenho, além da identificação da placa. Creio que seria melhor o autor localizar melhor o ambiente geotectônico descrito na figura, assim as forças terão mais sentido para o leitor.

O propósito do artigo é chamativo, no final da introdução o autor diz que o objetivo principal do trabalho é encontrar a associação entre queda de esforços, esforços de ambientes tectônicos e o esforços que controlam o movimento das placas litosféricas. A finalização e desfecho do artigo não me pareceu claro, pois o leitor não é levado a um pensamento convergente, nas conclusões não se alcança o objetivo prosposto no último parágrafo da introdução. A dificuldade de leitura do texto interferiu no entendimento total do mesmo, no entanto, creio que mesmo assim o autor não conseguiu comprovar de fato essa relação. Uma clareza na apresentação de dados seria necessária para refutar ou validar a relação entre estes tipos de esforços.

\bibliographystyle{seg.bst}
\bibliography{References.bib}
    
\end{document}
