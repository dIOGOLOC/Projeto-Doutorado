%\documentclass[paper,twocolumn]{geophysics}
\documentclass[manuscript]{geophysics}[10pt]
%\documentclass[long]{geophysics}
% An example of defining macros

\usepackage[portuguese]{babel}
\usepackage[T1]{fontenc}
\usepackage[utf8]{inputenc}



\usepackage{amsmath}


\newcommand{\rs}[1]{\mathstrut\mbox{\scriptsize\rm #1}}
\newcommand{\rr}[1]{\mbox{\rm #1}}
\usepackage{color}

\begin{document}

\begin{center}
\textbf{\LARGE
 Artigo 11 - Some Remarks on the Development of Sedimentary Basins - \cite{mckenzie_remarks_1978}} \\
\textit{Diogo Luiz de Oliveira Coelho}
\footnote{\textit{Universidade Federal do Rio Grande do Norte - UFRN
Centro de Ciências Exatas e da Terra - CCET
Departamento de Geofísica
Campus Universitário - Lagoa Nova
59072-970 Natal, RN}}
\end{center} 

A publicação realizada por \cite{mckenzie_remarks_1978} proprõe um modelo para formação de bacias sedimentares embasado na evolução termomecânica da região. O modelo consiste numa ideia revolucionária para a época, pois a inserção do estiramento na formação de bacias explica a presença de anomalia térmica na região da bacia sedimentar que nenhum outro tinha explicado de forma plausível. O modelo proposto por \cite{mckenzie_remarks_1978} consiste em um estiramento da litosfera continental que produz um afinamento da crosta e um soerguimento astenosférico. Tal soerguimento gera um subsidência a partir do esfriamento progressivo e recuo da astenosfera abaixo da bacia. Com a explicação da modelagem o autor prepara muito bem o caminho para a comprovação desse novo pensamento prosposto. O autor mostra que a subsidência e o fluxo de calor só dependem da quantidade de extensão da crosta, em função do tempo ocorrido neste processo de formação de bacia. Tudo isso fica muito bem explicado, até mesmo as linearizações e aproximações feita.

O autor afirma que o grau de estiramento pode ser medido a partir diferença entre a espessura crustal inicial e final. \cite{mckenzie_remarks_1978} na seção 3 utiliza-se de dados geológicos e geofísicos para comprovar seu modelo, porém a apresentação desses dados não ajuda de forma nenhuma, ao contrário, traz mais dúvidas à modelagem proposta. A teoria postulada pelo autor é muito bem feita e realmente engloba todas as questões em aberto na época sobre a formação de bacias, porém como mostrado na seção 3 do artigo, a comprovação através de dados geológicos e geofísicos não consegue mensurar a taxa de estiramento nas bacias sedimentares propostas, Great Basin e Mar do Norte, e por conseguinte, tal modelo não possui nenhuma comprovação com dados reais. Isso faz com que a teoria prosposta não ganhe crédito. Creio que na seção 3 o autor não teve dados suficientes para comprovar a modelagem, e por isso apresentação dos "dados" não contribuiu para o trabalho, na verdade trouxe a tona o descompasso entre a modelagem proposta e os dados observados.
Levando em consideração a pouca quantidade de dados sobre a estrutura crustal na década de 70 é aceitável a dificuldade de se constatar tal modelo. Tanto que com o passar dos anos, e com a facilidade de se obter mais dados em grande parte do globo, pôde-se comprovar que o modelo proposto por \cite{mckenzie_remarks_1978} foi o mais aceito para a explicação da evolução das bacias sedimentares. É claro que com o tempo este modelo foi sendo aprimorado em vários pontos, como a adição de um estiramento contínuo, a diferenciação da litosferas em várias camadas, etc. 

\bibliographystyle{seg.bst}
\bibliography{References.bib}
    
\end{document}
