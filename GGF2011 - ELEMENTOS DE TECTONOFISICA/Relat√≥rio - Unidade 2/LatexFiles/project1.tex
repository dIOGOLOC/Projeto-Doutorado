%\documentclass[paper,twocolumn]{geophysics}
\documentclass[manuscript]{geophysics}[10pt]
%\documentclass[long]{geophysics}
% An example of defining macros

\usepackage[portuguese]{babel}
\usepackage[T1]{fontenc}
\usepackage[utf8]{inputenc}



\usepackage{amsmath}


\newcommand{\rs}[1]{\mathstrut\mbox{\scriptsize\rm #1}}
\newcommand{\rr}[1]{\mbox{\rm #1}}
\usepackage{color}

\begin{document}

\begin{center}
\textbf{\LARGE
 Artigo 02 - Rises, Trenches, Great Faults, and Crustal Blocks - \cite{morgan_rises_1968}} \\
\textit{Diogo Luiz de Oliveira Coelho}
\footnote{\textit{Universidade Federal do Rio Grande do Norte - UFRN
Centro de Ciências Exatas e da Terra - CCET
Departamento de Geofísica
Campus Universitário - Lagoa Nova
59072-970 Natal, RN}}
\end{center} 

O autor considerou três tipos de limites entre os blocos litosféricos como mostrado na Figura 1. Lembrando que o artigo é de 1968 e não existiam informações de GPS e nem grandes redes sismográficas disponíveis para tal análise, pode-se considerar que o autor distingue as principais placas litosféricas com um grau de detalhe considerável. Como mostrado no artigo anteriormente estudado, dados do continente americano, parte leste da América do Norte e do Sul, e do Japão possuem um grau de detalhe maior que os demais, podendo ser pela quantidade de dados ou até mesmo por não ser tão complexo quanto a região dos Alpes e do Himalaia. O oceano Atlântico, comparado ao Pacífico, é bem delimitado, creio que isso é devido pela quantidade de estudos e pela facilidade em obter dados. Além da complexidade dos processos geológicos ocorridos no Atlântico ser menor que no Pacífico.

O artigo introduz as principais feições geotectônicas, explicando cada estrutura e exemplificando uma a uma. Isso foi importante para caracterizar cada bloco na figura 1. No entanto, o autor diminui a dinâmica do texto quando quando faz referências às figura utilizadas. O trabalho está bem ilustrado, porém o autor poderia condensar algumas figuras, como por exemplo as figuras 2 e 3 e as figuras 4,5 e 6. No texto o autor cita a figura 8 na mesma página que a figura 5 está apresentada, se houvesse a aglutinação dessas figuras, assim como outras, como também as figuras 7,8 e 10, o texto iria fluir mais. O mosaico apresentado na figura 14 também podia ser melhor organizado, pois o autor utiliza 3 páginas para comparar dados diferentes na mesma região. O autor poderia modificar a figura para duas colunas, assim facilitaria a comparação dos diferentes dados utilizados. Essa grande quantidade de páginas para uma mesma figura dificulta a apresentação das informações e a comparação dos vários tipos de dados.  

O movimento das placas litosféricas na superfície da Terra pode ser mensurado utilizando o teorema de Euler, como mostra \cite{morgan_rises_1968}. Este teorema mostra que o movimento relativo entre as placas é unicamente definido por uma separação angular ao redor de um polo de movimento relativo, chamado de pólo de Euler. Um aspecto importante do movimento relativo das placas é que o pole tende a permanecer invariante para longos períodos de tempo. As velocidades das placas são igualmente constantes durante um período de vários milhões de anos. O pólo de rotação das placas litosféricas pode ser determinado com a construção de grandes círculos de ângulo reto margeando o ponto de intersecção comum. O artigo apresenta uma base clara para utilização deste método, além de exemplificar e fundamentar cada resultado. A utilização do movimento relativo de placas adjacentes se faz necessário para calcular a velocidade de outras placas. Dados recentes calculados através de GPS mostram claramente que os movimentos calculados por \cite{morgan_rises_1968} se ajustam bem aos resultados obtidos com a tecnologia atual. Tanto que as conclusões apresentadas pelo autor tem um objetivo muito maior que apenas dissertar sobre a velocidade das placas, e sim gerar dados para ter uma base sólida sobre a teoria da tectônica de placas.

\bibliographystyle{seg.bst}
\bibliography{References.bib}
    
\end{document}
