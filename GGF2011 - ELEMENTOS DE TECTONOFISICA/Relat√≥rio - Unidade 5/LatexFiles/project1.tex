%\documentclass[paper,twocolumn]{geophysics}
\documentclass[manuscript]{geophysics}
%\documentclass[long]{geophysics}
% An example of defining macros

\usepackage[portuguese]{babel}
\usepackage[T1]{fontenc}
\usepackage[utf8]{inputenc}



\usepackage{amsmath}


\newcommand{\rs}[1]{\mathstrut\mbox{\scriptsize\rm #1}}
\newcommand{\rr}[1]{\mbox{\rm #1}}
\usepackage{color}

\begin{document}

\begin{center}
\textbf{\LARGE
 Artigo 05 - The Strength of Intraplate Lithosphere - \cite{kusznir_strength_1984}} \\
\textit{Diogo Luiz de Oliveira Coelho}
\footnote{\textit{Universidade Federal do Rio Grande do Norte - UFRN
Centro de Ciências Exatas e da Terra - CCET
Departamento de Geofísica
Campus Universitário - Lagoa Nova
59072-970 Natal, RN}}
\end{center} 

\cite{kusznir_strength_1984} possui um ponto fraco que é evidenciado ao final da leitura da introdução. Os objetivos e prospostas do artigo não ficam claros. O leitor não é informado sobre o que busca o artigo, o objetivo principal da pesquisa. Expor de maneira clara os objetivos dos trabalho é fundamental em um artigo científico. Também é necessário interar o leitor nos principais pontos a serem solucionados e o que este artigo acrescentará ao conhecimento atual na introdução. Não existe uma boa exposição das metas a serem atingidas pelo artigo, no entanto, no resumo do artigo existe uma definição, mesmo que modesta, do que será feito, mas seria necessário um parágrafo na introdução pra expor de maneira clara e explícita os objetivos do trabalho.

Uma dúvida sobre o objetivo proposto é que se será possível com as premissas feitas obter uma comparação razoável entre resistência da litosfera com os esforços intraplaca. O termo "comparação razoável" dá-se pelas hipóteses de ser uma deformação plana ($\epsilon _{y} = 0$) e esforços ascendendtes serem iguais a zero ($\sigma _{z} = 0$). A modelagem prosposta é simples, mas creio que para uma tectônica distensiva é aplicável e pode ser considerada plausível num dado objetivo. A modelagem apresentada representa de forma viável uma tectônica distensiva, um aperfeiçoamento do modelo teria melhores resultados, no entanto, para uma ótica regional o tipo de modelagem e premissas feitas conseguem elucidar o problema colocado. 

Os resultados e discussões apresentadas por \cite{kusznir_strength_1984} cumprem o papel de representar todo o potencial do trabalho, porém, como dito anteriormente, faltou uma melhor exposição dos objetivos do trabalho. Creio que as discussões foram bem embasadas tanto pelas tabelas quanto pelos gráficos. Vale a pena ressaltar que o trabalho foi bem ilustrado e isso ajudou bastante no entendimento dos dados. 

Pesquisando na internet pude encontrar que o artigo de \cite{kusznir_strength_1984} possui a mesma temática que os artigos \cite{kusznir_intraplate_a_1984} e \cite{kusznir_intraplate_b_1984}. Tais artigos foram publicados no mesmo ano pelo mesmo autor. No entanto, os artigos \cite{kusznir_strength_1984} e \cite{kusznir_intraplate_b_1984} foram pré-requisitos para a uma publicação de grande impacto (\cite{kusznir_intraplate_a_1984}), tal afirmação é feita pois estes artigos são subsídios para embasar a relação entre a resistência da litosfera intraplaca com o fluxo de calor. Por isso o artigo, creio eu, o artigo não traça um grande objetivo. Vale ressaltar que o artigo cumpriu o seu papel de passar a conteúdo proposto da melhor maneira. 

\bibliographystyle{seg.bst}
\bibliography{References.bib}
    
\end{document}
