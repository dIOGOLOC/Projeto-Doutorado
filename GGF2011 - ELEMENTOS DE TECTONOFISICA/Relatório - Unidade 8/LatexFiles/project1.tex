%\documentclass[paper,twocolumn]{geophysics}
\documentclass[manuscript]{geophysics}[10pt]
%\documentclass[long]{geophysics}
% An example of defining macros

\usepackage[portuguese]{babel}
\usepackage[T1]{fontenc}
\usepackage[utf8]{inputenc}



\usepackage{amsmath}


\newcommand{\rs}[1]{\mathstrut\mbox{\scriptsize\rm #1}}
\newcommand{\rr}[1]{\mbox{\rm #1}}
\usepackage{color}

\begin{document}

\begin{center}
\textbf{\LARGE
 Artigo 08 - The Motion of Thrust Sheets - \cite{elliott_motion_1976}} \\
\textit{Diogo Luiz de Oliveira Coelho}
\footnote{\textit{Universidade Federal do Rio Grande do Norte - UFRN
Centro de Ciências Exatas e da Terra - CCET
Departamento de Geofísica
Campus Universitário - Lagoa Nova
59072-970 Natal, RN}}
\end{center} 

\cite{elliott_motion_1976} tenta analisar todos os elementos envolvidos na formação de cinturões de cavalgamentos, como as relações e implicações das propriedades físicas neste tipo de ambiente. Segundo o autor, toda a teoria gerada busca ter o menor número de premissas possível, pois é uma tentativa de explicar de uma maneira mais realística os processos ocorridos. O autor também mostra que conhecendo somente a força de cisalhamento na zona de cavalgamento pode-se fazer um progresso considerável no conhecimeno sobre este tema. As conclusões geradas sobre involvem os processos tectônicos regionais atuantes neste tipo ambiente geotectônico, logo deixa-se em aberto os processos que ocorrem em escala local. Mas convenhamos que não tem como fazer um artigo explicando cada detalhe sobre um ambiente tectônico, principalmente um ambientet tão complexo quanto um sistema de cavalgamentos.

Na leitura do texto observa-se que existe para grande parte das premissas feitas uma discussão sobre as possíveis justificativas, como por exemplo, a negligência dos gradientes de esforços longitudinais em comparação com as forças induzidas gravitacionalmente. Foi mostrado que através da troca de esforços e da variação da direção de esforços, para grandes espessuras de H, as forças gravitacionais são predominantes.

As conclusões apresentadas pelo trabalho são muitas, e algumas são bem difícieis de se aceitar e até mesmo entender. \cite{elliott_motion_1976} conclui que não é necessário uma colisão continente-continente para a formação de ofiolítos, como exemplo ele cita o grande ofiolito de Omã. A formação de ofiolítos é ainda bem discutida atualmente, no entanto, o autor apenas informa que os ofiolítos são formados por esforços graviationais. Alguns trabalhos de sísmica foram citados pelo autor nesta seção, porém nenhum trabalho foi discutado para embasar tal conclusão. O que \cite{elliott_motion_1976} mostrou foram apenas algumas seções esquemáticas das áreas estudas, creio que a apresentação de dados geofísicos poderiam ajudar na ratificação da teoria prosposta. A confirmação deste tipo de conclusão é bem difícil de ser avaliada com os dados e teoria mostrados, creio que para confirmar seria um necessário mais apurado e com um grau de detalhamento maior. Além da necessidade de agregar mais dados geofísicos e geológicos, como sísmica, sismologia, gravimetria, magnetometria, magnetotelúrico e etc.

Para melhor exemplificar o fenômeno de formação dos cinturões de cavalgamento seria necessário explicar os mecanismos físicos que operam ao longo da superfície de falha durante o movimento, pois nesse artigo foram feitas considerações estáticas, e ainda por cima não se considerou as propriedades dos materiais circundantes. Desconsiderar os mecanismos físicos que operam durante o deslizamento da placa é negligenciar as pressão de poro e como a placa se comporta ao deslizar. Claro que para uma primeira aproximação é plausível este tipo de modelo simplificado e com o tempo a evolução do conhecimento sobre esse tipo de ambiente é normal.

\bibliographystyle{seg.bst}
\bibliography{References.bib}
    
\end{document}
