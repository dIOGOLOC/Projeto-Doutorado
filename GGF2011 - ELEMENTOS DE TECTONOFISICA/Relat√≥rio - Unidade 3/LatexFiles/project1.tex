%\documentclass[paper,twocolumn]{geophysics}
\documentclass[manuscript]{geophysics}
%\documentclass[long]{geophysics}
% An example of defining macros

\usepackage[portuguese]{babel}
\usepackage[T1]{fontenc}
\usepackage[utf8]{inputenc}



\usepackage{amsmath}


\newcommand{\rs}[1]{\mathstrut\mbox{\scriptsize\rm #1}}
\newcommand{\rr}[1]{\mbox{\rm #1}}
\usepackage{color}

\begin{document}

\begin{center}
\textbf{\LARGE
 Artigo 03 - Evolution of Triple Junction - \cite{mckenzie_evolution_1969}} \\
\textit{Diogo Luiz de Oliveira Coelho}
\footnote{\textit{Universidade Federal do Rio Grande do Norte - UFRN
Centro de Ciências Exatas e da Terra - CCET
Departamento de Geofísica
Campus Universitário - Lagoa Nova
59072-970 Natal, RN}}
\end{center} 

\cite{mckenzie_evolution_1969} define como junção tríplice o limite entre três placas litosféricas. Todas as possibilidades geométricas no encontro entre três placas é discutido e exemplificado ao longo do artigo, assim como a evolução e estabilidade das junções tríplices. O artigo possui uma introdução bem estruturada, bem referenciada e define cada feição utilizada no decorrer do artigo. No entanto, não achei clara a definição do objetivo do trabalho ao final da introdução. Creio que os autores poderiam gastar mais uma linha para realmente exclarecer e fixar o objetivo do artigo, pois "a consequência da movimentação de placas rígidas" não transmite a ideia do estudo da evolução temporal das junções tríplices. 

Os autores não elucidaram o fato de não existirem limites entre 4 ou mais placas litosféricas, apenas citaram o fato de que não existem tais pontos, exceto instantaneamente. Acho que seria válido um parágrafo explicando tal fato, pois pavimentaria a evolução das junções tríplices e deixaria o leitor mais atento à evolução temporal desse tipo de feição e sua importância na tectônica de placas. Com o passar do artigo o autor mostra que existem vaŕios tipos de junções tríplices, um total de 16, classificadas em estáveis e instáveis no tempo, tal tema foi muito bem abordado e exemplificado. Entretanto, uma dúvida surgiu no momento que o autor mostra a instabilidade das junções tríplices, pois se, mesmo que instantaneamente, possam existir junções com 4 ou mais placas, qual seria a diferença no tempo geológico entre essas junções com as junções tríplices instáveis? Como já dito anteriormente, creio que seria necessário mais exclarecimento nessa parte para deixar o leitor ciente de como é a evolução temporal dessas feições. E se o tempo de existência de junções com mais de 3 placas é considerável ou não. Os movimentos das placas são denominados geologicamente instantâneos, e se referem a movimentos médios durante um período geológico muito curto. Em sala de aula foi explicado que no antes da movimentação das placas rígidas poderiam existir limites com mais de 3 placas, no entando após essa movimentação esse limites migrariam para junções tríplices, no entando a questão que fica é: será que esse tempo é descondirável em relação ao tempo que permanecem as junções tríciples instáveis? Por isso acho que este ponto merece ser melhor abordado.

\bibliographystyle{seg.bst}
\bibliography{References.bib}
    
\end{document}
