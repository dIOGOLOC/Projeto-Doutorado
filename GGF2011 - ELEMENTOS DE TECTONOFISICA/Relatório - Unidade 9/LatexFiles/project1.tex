%\documentclass[paper,twocolumn]{geophysics}
\documentclass[manuscript]{geophysics}[10pt]
%\documentclass[long]{geophysics}
% An example of defining macros

\usepackage[portuguese]{babel}
\usepackage[T1]{fontenc}
\usepackage[utf8]{inputenc}



\usepackage{amsmath}


\newcommand{\rs}[1]{\mathstrut\mbox{\scriptsize\rm #1}}
\newcommand{\rr}[1]{\mbox{\rm #1}}
\usepackage{color}

\begin{document}

\begin{center}
\textbf{\LARGE
 Artigo 09 - On the Regional Variation of Heat Flow, Geotherms, and Lithospheric Thickness - \cite{pollack_regional_1977}} \\
\textit{Diogo Luiz de Oliveira Coelho}
\footnote{\textit{Universidade Federal do Rio Grande do Norte - UFRN
Centro de Ciências Exatas e da Terra - CCET
Departamento de Geofísica
Campus Universitário - Lagoa Nova
59072-970 Natal, RN}}
\end{center} 

O objetivo do trabalho é proposto na primeira frase do artigo, utilizar a variação do fluxo de calor global para determinar as geotermas e a espessura da litosfera para todas as regiões do planeta. Os autores utilizaram o campo de fluxo de calor global calculadoo através do preditor empírico de \cite{chapman_global_1975} como 
utilizar como principal limitador para calcular as geotermas. E com as geotermas calculadas é possível calcular o mapa de espessura da litosfera para o planeta. O trabalho de \cite{pollack_regional_1977} está muito bem fundamentado e apresenta uma excelente linha de raciocício lógico. A seguir serão destacados alguns pontos onde o trabalho faltou esclarecimento ou emitiu conclusões sem muito fundamento.

O artigo consegue fazer a transição entre os tópicos e consegue mostrar claramente a medologia aplicada. Principalmente algumas nuances metodológicas para as crostas oceânica e continental. No entanto, o trabalho apresenta seus resultados em figuras faltando informações geográficas. Existe uma grande discussão de resultados, porém conseguir identificar as provincías tectônicas nas figuras 1 e 6 é bem complicado, pois você não sabe qual a base cartográfica que os mapas foram gerados, qual é o meridiano de referência e qual a projeção cartográfica apresentada. Os autores deviam colocar pelo menos os limites dos países para que o leitor consiga localizar na figura as informações citadas nas discussões.

No final do artigo, ao meu ver, as conclusões fazem associações muito além do que os resultados obtidos. Existe uma grande discussão em relação a movimentação das placas litosféricas, todavia os autores não apresentaram nenhum dado ou cálculo que sinalizem isso. Com apenas o cálculo das espessuras, principalmente com uma distribuição irregular de dados, não produz um embasamento para algumas colocações propostas. O autor cita a equação 2 para mostrar a razão da deformação devido a vários processos no manto superior, porém o autor não apresenta resultados e sim uma associação com a espessura da litosfera encontrada. Nessa parte o autor ao invés de fazer tal associação, poderia gerar outra tabela com estes valores em regiões de escudo e em regiões oceânicas e com isso poderia discutir melhor e apresentar conclusões mais embasadas.


\bibliographystyle{seg.bst}
\bibliography{References.bib}
    
\end{document}
