%\documentclass[paper,twocolumn]{geophysics}
\documentclass[manuscript]{geophysics}[10pt]
%\documentclass[long]{geophysics}
% An example of defining macros

\usepackage[portuguese]{babel}
\usepackage[T1]{fontenc}
\usepackage[utf8]{inputenc}



\usepackage{amsmath}


\newcommand{\rs}[1]{\mathstrut\mbox{\scriptsize\rm #1}}
\newcommand{\rr}[1]{\mbox{\rm #1}}
\usepackage{color}

\begin{document}

\begin{center}
\textbf{\LARGE
 Artigo 01 - A New Class of Faults and Their Bearing on Continental Drift - \cite{wilson_new_1965}} \\
\textit{Diogo Luiz de Oliveira Coelho}
\footnote{\textit{Universidade Federal do Rio Grande do Norte - UFRN
Centro de Ciências Exatas e da Terra - CCET
Departamento de Geofísica
Campus Universitário - Lagoa Nova
59072-970 Natal, RN}}
\end{center} 


Fazer uma crítica sobre um artigo tão imponente e marcante para a geologia e geofísica global é complicado. O artigo de \cite{wilson_new_1965} apresenta um formato para uma revista, logo não há uma separação de seções em si. Este fato faz com que o artigo não flua tão bem na hora de ler, pois existem mudanças abruptas e também por se tratar de uma grande compilação de trabalhos apresentados em dois simpósios e de artigos e livros sobre "continental drift". Vê-se essa dificuldade pois é um artigo de cunho informativo e que visa a classificação de grandes zonas de falhas ao redor do mundo. 

Fazer uma crítica para um trabalho de 1965 é bem complicado, pois a quantidade de dados disponíveis hoje não condiz com a realidade da época. Principalmente quando comparamos a capacidade computacional da época. No entando, o autor no primeiro parágrafo cita a presença de uma sismicidade alta em zonas de falhas e nas cordilheiras meso-oceânicas, porém não cita fonte alguma, e acho que seria importante fontes para confirmar as afirmações feitas. O autor não precisaria adicionar outra figura, pois o trabalho está bem ilustrado, e sim, complementar com dados geofísicos/sismológicos a figura 1, onde mostra o globo terrestre com os principais limites tectônicos. Tais dados poderias ser a atividade sísmica, vulcanismo, dentre outros. Ou o autor pode ter escolhido uma figura mais enxuta, mostrando apenas as principais feições.

Um problema que surgiu ao ler o artigo foi não conseguir localizar algumas informações que o autor cita no texto , como por exemplo ao falar da figura 9, o autor citar a falha transformante de "\textit{British Columbia}". O leitor que não é familiarizado com a geografia do Canadá, de onde o autor é oriundo, não consegue entender bem a informação, já que na figura a simbologia de falha transformante não está bem marcada. Esste tipo de falta de referência dificulta o entendimento e a exposição dos dados. Creio que se algumas figuras do artigo fossem melhor referenciadas geograficamente melhoraria a mensagem passada, como por exemplo a figura 5 apresenta informações que localizam espacialmente o leitor no mapa mostrado, já as figuras 8 e 9 precisariam demais informações geográficas, como nome de algumas cidades ou estados dos EUA e Canadá.

Neste artigo J. Tuzzo Wilson exemplifica 6 tipos de falhas com movimento dextral e mostra que essas falhas transformantes são fundamentais para postular a deriva continental. Além disso cita que além dos 6 tipos de falhas dextrais existem as falhas com movimentos sinistrais. Creio que para uma compilação de dados geológicos e geofísicos este trabalho é bem claro e exemplifica com grande clareza e ditática os limites de placas mais estudadas do planeta. Pois a falha de Santo André, a abertura do chifre da África e do Atlântico e a zona de subducção da América do Sul são exemplos bem ditáticos da deriva continental. 



\bibliographystyle{seg.bst}
\bibliography{References.bib}
    
\end{document}
