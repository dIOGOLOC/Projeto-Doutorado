%\documentclass[paper,twocolumn]{geophysics}
\documentclass[manuscript]{geophysics}
%\documentclass[long]{geophysics}
% An example of defining macros

\usepackage[portuguese]{babel}
\usepackage[T1]{fontenc}
\usepackage[utf8]{inputenc}



\usepackage{amsmath}


\newcommand{\rs}[1]{\mathstrut\mbox{\scriptsize\rm #1}}
\newcommand{\rr}[1]{\mbox{\rm #1}}
\usepackage{color}

\begin{document}

\begin{center}
\textbf{\LARGE
 Artigo 06 - On the stress system that formed the Laramide Wind River Mountains, Wyoming - \cite{brewer_stress_1980}} \\
\textit{Diogo Luiz de Oliveira Coelho}
\footnote{\textit{Universidade Federal do Rio Grande do Norte - UFRN
Centro de Ciências Exatas e da Terra - CCET
Departamento de Geofísica
Campus Universitário - Lagoa Nova
59072-970 Natal, RN}}
\end{center} 

\cite{brewer_stress_1980} apresenta um artigo curto, logo espera-se um artigo conciso, bem escrito e direto, porém a introdução está difusa e o objetivo do trabalho não aparece na introdução. Este objetivo é visível apenas no \textit{abstract}, mesmo assim não fica tão claro o que foi feito no artigo após ler a introdução do artigo. Ao final da introdução seria preciso esclarecer o objetivo principal do artigo, além de falar um pouco sobre os resultados obtidos no trabalho. A melhoria da introdução se faz necessário pois não existe uma transição clara dos pontos abordados no artigo.

Apesar de uma introdução complicada, o restante do trabalho possui uma transição suave entre metodologia, resultados e concluções, trazendo uma fluidez na leitura e no entendimento do artigo. Trata-se de uma aplicação em dados reais da teoria de Anderson, para a partir dai calcular o o coeficiente de fricção. Mas o que realmente chama atenção nesse pequeno artigo é o coeficiente de fricção calculado. Pois o autor calcula um coeficiente de 0.36, um valor muito abaixo do calculado em experimentos laboratoriais, por volta de 0.6 - 0.8. Entendo que possa haver erros devido às diferentes condições entre o laboratório e o campo, porém o valor observador de duas a três vezes menor que o valor previsto em laboratório não é fácil de digerir. Mesmo apresentando algumas alternativas para um baixo valor encontrado, como a presença de minerais de argila (filossilicatos) e a diferença de escala, não é palpável essa grande diferença entre o valor previsto e o observado. Ou seria necessário rever a metodologia dos experimentos laboratoriais realizados ou realmente adicionar mais variáveis de controle às análises laboratoriais.

As figuras dispostas no artigo necessitam de melhorias pontuais na visualização da linha sísmica, principalmente em relação aos nomes das estações e de alguns nomes de áreas, e na seção sísmica é bem complicado observar todas as estruturas delimitadas na seção interpretada. Seria interessante anexar a figura 2 uma seção geológica para poder gerar uma melhor visualização da seção sísmica.

\bibliographystyle{seg.bst}
\bibliography{References.bib}
    
\end{document}
