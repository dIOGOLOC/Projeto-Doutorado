%\documentclass[paper,twocolumn]{geophysics}
\documentclass[manuscript]{geophysics}
%\documentclass[long]{geophysics}
% An example of defining macros

\usepackage[portuguese]{babel}
\usepackage[T1]{fontenc}
\usepackage[utf8]{inputenc}



\usepackage{amsmath}


\newcommand{\rs}[1]{\mathstrut\mbox{\scriptsize\rm #1}}
\newcommand{\rr}[1]{\mbox{\rm #1}}
\usepackage{color}

\begin{document}

\begin{center}
\textbf{\LARGE
 Artigo 04 - Some Simple Physical Models for Absolute Plate Motions - \cite{solomon_simple_1974}} \\
\textit{Diogo Luiz de Oliveira Coelho}
\footnote{\textit{Universidade Federal do Rio Grande do Norte - UFRN
Centro de Ciências Exatas e da Terra - CCET
Departamento de Geofísica
Campus Universitário - Lagoa Nova
59072-970 Natal, RN}}
\end{center} 

\cite{solomon_simple_1974} apresenta uma introdução bem curta sobre a literatura do tema abordado. No entanto, os autores deixam bem claro o objetivo do trabalho e como será abordado o problema, no caso o movimento absoluto das placas litosféricas. Para calcular o movimento absoluto de cada placa o autor precisa necessariamente limitar as placas litosféricas sobre a superfície da terra, no entando o autor não explica e nem referencia muito bem esta parte. Falta informação na delimitação dos limites das placas, o autor escreve que utiliza dados sísmológicos para delimitar os limites de placas tectônicas mundial, porém não referencia e nem cita as fontes em que esse tipo de dado foi retirado, apenas cita a referência para o limite entre a placa americana e da Eurásia.

Uma detalhe do artigo que os autores poderiam melhorar seria a exposição dos resultados. As figuras 2,3,4 e 5 mostram as velocidades absolutas obtidas pelos modelos A,B,C e D. Creio que para comparar os 4 diferentes modelos o autor poderia fazer um mosaico com as figuras, pois as figuras não apresentam uma grande quantidade de dados e podem ter o tamanho diminuído para uma melhor comparação. Pois tais figuras tem como base os limites das placas tectônicas mostrados na figura 1. Como o formato é o mesmo para todas as figuras, um mosaico é o melhor jeito para representar as velocidades absolutas calculadas para cada modelo de forças. Isto também ajudaria muito no texto, pois facilitaria na citação das figuras. Quanto mais figuras dispersas no texto mais complicado para o leitor se referenciar no texto.

No final do artigo os autores sumarizam os tópicos mais importante levantados durante o  decorrer do texto, uma espécie de conclusão. A divisão das seções no artigo não contribuiu para levar a mensagem da melhor maneira, pois ao final os autores precisaram fazer um apanhado geral. Creio que o formato padrão (introdução, metodologia, resultados e discussões) ajuda bastante o leitor na compreensão e na absorção do texto apresentado. Já no artigo observa-se que os tópicos não demonstra uma sequência simples para acompanhar. Já no artigo observa-se que os tópicos não demonstram uma transição suave entre si. Devido a isso as informações importantes ficam espalhadas pelo texto, fazendo o autor se preocupar sumarizar as principais informações no final. Isso mostra que independentemente do tamanho do artigo o jeito que se escreve e a estruturação do texto contribui bastante para a leitura. 

\bibliographystyle{seg.bst}
\bibliography{References.bib}
    
\end{document}
