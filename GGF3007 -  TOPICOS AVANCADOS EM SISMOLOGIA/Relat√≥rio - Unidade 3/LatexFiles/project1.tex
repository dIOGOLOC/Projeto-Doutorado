\documentclass[paper,twocolumn]{geophysics}
%\documentclass[manuscript]{geophysics}
%\documentclass[long]{geophysics}
% An example of defining macros

\usepackage[portuguese]{babel}
\usepackage[T1]{fontenc}
\usepackage[utf8]{inputenc}



\usepackage{amsmath}


\newcommand{\rs}[1]{\mathstrut\mbox{\scriptsize\rm #1}}
\newcommand{\rr}[1]{\mbox{\rm #1}}
\usepackage{color}

\begin{document}

\title{{UNIDADE 03 - Ondas de Corpo (02/05/2016 - 18/05/2016)}}
%\renewcommand{\thefootnote}{\fnsymbol{footnote}} 

\address{\footnotemark[1] Universidade Federal do Rio Grande do Norte - UFRN
Centro de Ciências Exatas e da Terra - CCET
Departamento de Geofísica
Campus Universitário - Lagoa Nova
59072-970 Natal, RN
}
\author{Diogo Luiz de Oliveira Coelho\footnotemark[1]}
\lefthead{Coelho, DLO}
\righthead{UNIDADE 03 - Ondas de Corpo}

%\maketitle is unneccessary

\begin{abstract}
{Esta unidade visa aprofundar no processamento dos registros de ruído sísmico para a reconstrução da EGF de ondas de corpo, assim como na origem do ruído sísmico que permite essa reconstrução.}  
\end{abstract}

\section*{Introdução}
Na sísmica de exploração, reflexões de ondas de corpo são de interesse, pois provém imagens da subsuperfície de alta resolução. Principalmente quando se trata de  potenciais estruturas geradores/armazenadoras de hidrocarbonetos. Por isso após a descoberta e utilização do ruído sísmico ambiental, iniciou-se uma busca para refinar e entender os processos causadores desse sinal. Além disso, inúmeros trabalhos começaram a surgir para exeplorar o potencial da interferometria de reflexão, como mostra \cite{wapenaar_tutorial_2010}. Entretanto, a extração das reflexões das ondas de corpo utilizando a interferometria de ruído sísmico é muito mais complicada que a extração de ondas de superfície. 

Alguns fatores que geram essa dificuldade são os longos caminhos entre os receptores, a alta taxa de espalhamento geométrico, a dificuldade de fazer uma distribuição correta dos receptores, como mostra \cite{wapenaar_retrieving_2004} tais receptores deveriam estar em subsuperfície para recuperar essas ondas de corpo, e também o fato do ruído sísmico ambiental ser dominado por ondas de superfície. Por isso grande parte do trabalhos até agora já feitos sobre ruído sísmico ambiental tem como tema principal a emersão das ondas de superfície no ruído sísmico.

As ondas de corpo ainda estão em processo de estudo e de esclarecimento, principalmente no que tange a sua geração. A metodologia para a recuperação desse tipo de sinal ainda não está bem clara. A faixa de frequência em que as ondas de corpo são observadas, frequência relativamente baixas,  remetem à ondas excitadas pelos microssismos oceânicos. No entando a frequência de interesse  para sísmica de exploração é maior que 1 Hz, que na maioria das vezes são de difíceis de se encontrar \citep{wapenaar_tutorial_2010}.

Apesar das dificuldades de se recuperar as ondas de corpo no registro contínuo do ruído sísmico ambiental vários trabalhos tem surgido com este tema nos ultimos anos, como os trabalhos de \cite{roux_p-waves_2005}, \cite{prieto_amplitude_2011}, \cite{poli_emergence_2012} e \cite{boue_teleseismic_2013}. Cada um desses trabalhos apresentam uma aplicação diferente para as ondas de corpo oriundas do ruído sísmico. Por isso objetivo dessa unidade é aprofundar os conhecimentos no processamento dos registros de ruído sísmico para a reconstrução a função empírica de Green de ondas de corpo. 

\section*{Ondas de Corpo no Ruído Sísmico Ambiental}

O ruído sísmico ambiental é composto majoritariamente de ondas de superfície, como pode ser observado pela grande quantidade de artigos oriundos deste tema após os artigos de \cite{campillo_long-range_2003} e \cite{shapiro_emergence_2004}. Entretanto observa-se que na composição do ruído sísmico ambiental existe uma parcela que é oriunda de ondas de corpo, como mostra \cite{roux_p-waves_2005} e \cite{poli_emergence_2012}. \cite{roux_p-waves_2005} mostram tempos de percursos distintos para o sinal emergente do ruído sísmico ambiental, então pode-se discriminar dois tipos de sinais dentro do registro. Através das análise tempo-frequência pode-se identificar um sinal de alta frequência não-dispersivo, e que após a análise do movimento da partícula identificou uma polarização típica de onda de corpo.

Vimos na unidade passada que o ruído sísmico é composto principalmente por ondas de superfície, e as principais fontes são creditadas a pertubações de pressão na atmosfera e no oceano devido a similaridade da forma do espectro da pressão de fundo oceânico e do espectro do movimento da partícula ruído. É necessário lembrar que as ondas de superfície são predominantes no sinal, isso se torna marcante quando as fontes estão distribuídas próximas à superfície. No entanto, quando as fontes estão em profundidade, como mostrado em \cite{wapenaar_tutorial_2010}, é possível recuperar o sinal das ondas de corpo, através da interferometria de reflexão. Existe uma necessidade de uma estudo aprofundade sobre como são geradas as ondas de corpo presentes no ruído sísmico ambiental. \cite{gerstoft_global_2008} e \cite{gualtieri_modelling_2014} mostram que as ondas de corpo no ruído sísmico ambiental possue uma fonte distinta das ondas de superfície. Este autores mostram que essas ondas são geradas em tempestades em aǵuas profundas, tal interpretação é feita pela identificação do local da fonte, através do método \textit{beamforming}, sendo correlacionado com grandes tempestades marinhas. Outra fonte para as ondas de corpo é a heterogeneidade  local, que está diretamente ligada com o espalhamento. \cite{larose_fluctuations_2008} mostra o papel do espalhamento na geração do ruído sísmico ambiental, este tem um papel importante porque o primeiro trabalho de interferometria de ruído ambiental, \cite{campillo_long-range_2003}, está baseado neste princípio.

\cite{forghani_underestimation_2010} apresenta resultados numéricos para mostrar o motivo das ondas P não serem subestimadas nos registros do ruído sísmico ambiental quando comparadas com as ondas de superfície. Uma razão seria a má distribuição das fontes de ruído, uma distribuição não-homogênea de fontes ao redor dos receptores causam essa baixa representatividade das ondas de corpo nos registro. Para reconstruir as ondas de corpo é necessário um grande número de fontes estacionárias na região do levantamento sísmico, então em uma região onda exista um número limitado de fontes não será fácil recuperar o sinal das ondas de corpo no ruído sísmico ambiental. Essa característica é notada no trabalho de \cite{vidal_retrieval_2014}, porque logo após o processamento dos dados restaram um número ínfimo de registros de fontes geradoras de ondas de corpo comparados ao número total do levantamento. Quando as fontes estão na superfície, as correlações cruzadas das ondas de corpo contém um produto de ondas refletidas, isso faz com que as amplitudes do sinal sejam menores que as amplitudes verdadeiras das ondas de corpo refletidas. Para reconstruir a amplitude verdadeira das ondas de corpo é necessário que as fontes estejam em profundidade, como mostrado em \cite{wapenaar_tutorial_2010}. Todavia é difícil ter um número grande de fontes em subsuperfície e ter controle deste tipo de fonte, como mostra \cite{vidal_retrieval_2014}. Logo as ondas de corpo são subestimadas nas correlações cruzadas do ruído sísmico ambiental. 

\section*{Papel do Espalhameno no Ruído Sísmico Ambiental}

\cite{hennino_observation_2001} e \cite{larose_fluctuations_2008} estudaram qual é o papel do espalhamento na reconstrução da função de Green, através de registros de ondas Coda e de uma modelagem de de múltiplos espalhadores, respectivamente. A caracterização do espalhamento é dada como importante para o autores, pois pode-se ter um entendimento melhor da propagação das ondas e utilizar com mais coerência e esclarecimento informações obtidas pelas ondas Coda.

O primeiro trabalho a utilizar a correlação cruzada ruído sísmico oriundo de ondas Coda foi o artigo de \cite{campillo_long-range_2003}. Neste trabalho utiliza-se como campo de ruído sísmico os registros de ondas Coda, porém aconteceram algumas falhas na utilização desse tipo de sinal. \cite{hennino_observation_2001} cita dois bons motivos para se estudar a fundo o espalhamento: primeiro, pode-se extrair corretamente informações dos registros das ondas Coda, isto é, fazer o problema inverso; segundo, esse esclarecimento poderia facilitar a utilização de poderosas ferramentas mesoscópicas desenvolvidas em óptica e acústica. \cite{hennino_observation_2001} tenta entender as razões de energia calculadas através da equação da densidade de energia da onda elástica total assumindo que a Coda é campo de onda aleatório equiparticionado. A equipartição causada por espalhamento múltiplo tem uma propriedades notável que é ser independente de flutuações descolhecidas causadas pelo espalhamento, logo há uma independência entre a razão de energia com a magnitude, polarização e distância entre a estação e o evento. Atráves dos resultados obtidos pode-se constatar que a Coda sísmica é um processo de espalhamento múltiplo.

Através de uma grande modelagem com inúmeras fontes e espalhadores, \cite{larose_fluctuations_2008} demonstra como a onda direta gerada é influenciada pela presença de múltiplos espalhares ao redor dos dois receptores. Há uma série de simulações para tentar entender como funciona essa interação do sinal com os múltiplos caminhos percorridos pela onda direta. Neste artigo \cite{larose_fluctuations_2008} apresenta que trabalhos  teóricos e experimentais que mostram que espalhamento múltiplo tem um papel central na simetria no tempo e no espaço das correlações, e que até a convergência das correlações é afeta. Porém existe uma relação entre os espalhadores e a absorção do meio, portanto o autor coloca esse ponto em consideração. Pois um grande número de espalhadores próximo dos receptores pode causar uma atenuação do sinal registrado.
 
\section*{Metodologia aplicada}

A metodologia aplicada na reconstrução da Função de Green Empírica para as ondas de corpo segue os mesmos passos propostos por \cite{bensen_processing_2007}. Entretanto algumas etapas deste processamento diferem bastante nos trabalhos apresentados nesta unidade. O principal motivo para isso é o objetivo principal de cada trabalho. Por exemplo, \cite{prieto_amplitude_2011} necessita da informação da amplitude original dos sismogramas, logo nenhum filtro temporal foi aplicado aos registros, pois as normalizações temporais geram uma distorção nas amplitudes dos sismogramas. 

A seguir serão discutidos as etapas de processamento para a reconstrução da função empírica de green para  ondas de corpo. Tendo como base os trabalhos de \cite{roux_p-waves_2005}, \cite{prieto_amplitude_2011}, \cite{poli_emergence_2012} e \cite{boue_teleseismic_2013}. As divergências entre as metodologias abordadas serão discutidas em cada tópico. Tais tópicos são: Preparação do banco de dados, Correlação Cruzada e Processamento da função empírica de Green.

\subsection*{Preparação do banco de dados}

O banco de dados utilizados para recuperar as ondas de corpos nos trabalhos de \cite{roux_p-waves_2005},  \cite{poli_emergence_2012} e \cite{boue_teleseismic_2013} é composto por um grande número de estações de banda larga. Já o trabalho de \cite{prieto_amplitude_2011} utilizou diversos tipos de arranjos sísmicos para recuperar a informação da amplitude do campo sísmico ambiental, tanto para recuperar reposta do impulso em um edifício quanto para informações sobre bacias sedimentares.

\cite{roux_p-waves_2005} utiliza em seu banco de dados estações sismográficas localizadas em uma área de 11 $km^{2}$ na região da falha de San Andres. O tempo total de aquisição utilizado neste trabalho foi de 30 dias.  Já \cite{poli_emergence_2012} monta seu banco de dados com estações com distâncias variando aproximadamente de 50 a 600 km. Os dados ficaram 2 anos sendo adquiridos, entre 2007 e 2008, porém foi utilizado apenas um ano de resgistro para integrar o banco de dados. O trabalho mostrado por \cite{boue_teleseismic_2013} também utiliza um ano de registros, porém as estações estão espalhadas por todo o globo terrestre. 

Como mostrado em \cite{bensen_processing_2007}, inicialmente prepara-se os dados da forma da onda individulmente em cada estação. \cite{roux_p-waves_2005} mostra o mesmo procedimento removendo a média, tendência e resposta instrumental de cada estação. \cite{boue_teleseismic_2013} segue o mesmo procedimento que foi utilizado por \cite{poli_body-wave_2012}, que no fundo é semelhante ao procedimento inicial de \cite{poli_emergence_2012}, o qual retirou de \cite{bensen_processing_2007} as principais etapas. \cite{prieto_amplitude_2011} não exemplifica como procedeu nessa etapa inicial, porém não deve fugir do que foi feito, pois esse proceedimento inicial não afeta o objetivo de seu trabalho.

A metodologia utilizada para reduzir os efeitos de terremotos nesta unidade é bastante diversificada, pois alguns autores utilizam trucam janelas de sinal de tamanhos distintos para reduzir o efeito tanto de terremotos como de picos espúrios. Assim como para amenizar as irregularidades instrumentais e fontes de ruídos não-estacionários próximos á estação. \cite{roux_p-waves_2005}, \cite{poli_emergence_2012} e \citep{boue_teleseismic_2013} utilizam algo parecido a média-móvel dada por \cite{bensen_processing_2007}. Em uma janela de 4 horas de registro é calculada a média da janela, e qualquer sinal que seja maior que três vezes o desvio padrão do sinal da janela é removido. \cite{prieto_amplitude_2011} não faz uso de nenhuma normalização temporal nem frequencial, pois este autor quer preservar ao máximo a amplitude do sinal original.

Como abordado em \cite{bensen_processing_2007}, o ruído sísmico não é branco no domínio da frequência, então é necessário uma normalização espectral para alargar a banda do sinal e para diminuir a degradação causada por fontes persistentes. \cite{roux_p-waves_2005}, \cite{poli_emergence_2012} e \citep{boue_teleseismic_2013} aplicam um branqueamento espectral com as seguintes bandas de frenquências, respectivamente, 0.1 – 1.3 Hz, 0.1 - 2 Hz e 0.1 Hz - 0.5 Hz. Este branqueamento é para eliminar o sinal das fontes de microssismos primários e secundários, que são bem marcantes no espectro do sinal. \cite{larose_fluctuations_2008} cita que não fez uso do branqueamento para não modificar o conteúdo frequencial dos registros. 


\subsection*{Correlação Cruzada}

Depois de fazer as correções em cada estação separadamente, computa-se as correlações cruzadas entre todas as componentes do sensor, radial, vertical e transversal. Um fato que chama bastante atenção é que para recuperar as ondas de corpo no sinal oriundo do ruído sísmico, a janela de sinal utilizada para fazer as correlações é menor que a utilizada para ondas de superfície. Exceto o trabalho de \cite{roux_p-waves_2005} que utiliza o tamanho de 24 horas de sinal, todos os outros utilizam janelas menores. \cite{poli_emergence_2012} e \citep{boue_teleseismic_2013} utilizam uma janela de 4 horas para fazer a correlação cruzada, porém \citep{boue_teleseismic_2013} utiliza apenas a componente vertical. \cite{prieto_amplitude_2011} propõe uma metodologia nova, como visto anteriormente, sem nenhuma normalização temporal ou frequencial. Isso deve-se, segundo o autor, ao tamanho da janela escolhida para fazer as correlações cruzadas, de 1 hora no caso. Pois com uma janela muito pequena o efeito de terremotos ou picos espúrios é suprimido.

Após o cálculo das correlações cruzadas faz-se o empilhamento dessa janelas com o tempo total de registro. O tamanho da série temporal irá depender da velocidade das ondas e da distância entre as estações, e é importante frisar que quanto maior o número de correlações, maior a quantidade de sinal coerente empilhada. Há uma diferença marcante entre o processo de empilhamento feito por \citep{boue_teleseismic_2013}. Na unidade passada todo o empilhamento foi feito de forma linear, como também é feito por \cite{roux_p-waves_2005},\cite{prieto_amplitude_2011} e  \cite{poli_emergence_2012}. \citep{boue_teleseismic_2013} faz a correção \textit{move-out} e após faz o empilhamento  do tipo \textit{slant}, como mostrado no artigo de \citep{poli_body-wave_2012}. \cite{prieto_amplitude_2011} mostra algumas vantagens de seu método em relação ao método que é baseado diretamente em \cite{bensen_processing_2007}. Melhorias na qualidade do sinal, tanto na preservação da amplitude quanto na supressão de sinais de terremotos após o empilhamento é mostrado em seu artigo. Ele exemplifica que janela com uma hora de duração o sinal não sofre muitas distorções e apresentam informações valiosas sobre a amplitude, e isto pode ser utilizado para trabalhos sobre a atenuação.

Todos os autores mostram o sinal de ondas de corpo emergindo das correlações cruzadas, tanto entre as componentes radiais (R-R) quanto verticais (Z-Z). Pode-se observar em \cite{poli_emergence_2012} o sinal tanto da onda P quanto da S, porém a onda P aparece nas correlações entre as componentes radiais e a onda S entre as componentes verticas. Porém, como mostra \cite{boue_teleseismic_2013}, nas correlações entre as componentes verticais também pode ser encontrado as ondas P, S e suas demais fases. Já \cite{roux_p-waves_2005} consegue discriminar as ondas P em meio a correlações das componentes verticais. 

Se as fontes do ruído ambiental são distribuídas homegeneamente em todas as direções, a parte causal e acausal deveriam ser idênticas. No entanto, assimetrias são observadas nas amplitudes entre a parte causal e acausal. \cite{roux_p-waves_2005} utiliza esse informação das fontes de ruído sísmico como artifício para selecionar as correlações cruzadas que estão limitadas entre os azimutes de 40 a 60 graus. O autor seleciona esse azimute atráves \textit{Plane wave beamforming} que consegue identificar a distribuição angular da fonte de ruído e seleciona as correlações alinhadas com estas direções. \cite{poli_emergence_2012} também segue o mesmo princípio de alinhar as correlações cruzadas com as possíveis fontes de ruído. Este autor mostra que a as ondas de corpos são extremamente assimétricas nas correlações cruzadas, tanto na componente Radial quanto na Vertical. Tal assimetria mostra que sua fonte é diferente das ondas de superfície, pois estas aparecem simétricas em grande parte do registro. 

Nos trabalhos de de \cite{roux_p-waves_2005}e \cite{poli_emergence_2012} mostram as características das ondas de corpo recuperadas através das correlações cruzadas, tais características são: ondas não-dispersivas, ondas de frequência mais alta, e movimento da linear particula. Esses dados também foram confirmados com sinais oriundos de terremotos para ter uma maior confiança no resultado gerado, isso é o que mostra, \cite{prieto_amplitude_2011}, \citep{poli_emergence_2012} e \citep{boue_teleseismic_2013}. 

\section*{Resultados e Discussões}

As reconstruções da função de Green empírica para as ondas de corpo através da correlação do ruído sísmico mostrou-se mais complicada que para as ondas de superfície. A dificuldade inicial também é oriunda de que esse tipo de metodologia é nova quando comparada à outra. No entanto, existem muitas técnicas já desenvolvidas, como mostra \cite{wapenaar_tutorial_2010} em seu tutorial. Vários trabalhos já trazem uma certa formulação de uma metodologia robusta para recuperar esse tipo de onda tanto em escala regional quanto global, como visto nos trabalhos de \cite{roux_p-waves_2005} e \cite{boue_teleseismic_2013}, respectivamente.

Uma característica interessante é a capacidade de recuperar as funções de Green para ondas de corpo com arranjos distintos, escalas diferentes e processamentos bem divergentes, como o processamento de \cite{prieto_amplitude_2011}. Os resultados gerados por esses diversos trabalhos geram imagens tanto de atenuação de construções como de camadas mais profundas da terra, como são os trabalhos de \cite{prieto_amplitude_2011} e \cite{boue_teleseismic_2013}, respectivamente.

\section*{Conclusões}

Com o fim desta unidade pode-se entender um pouco melhor sobre como os registros do ruído sísmico pode reconstruir o sinal da EGF das ondas de corpo. A discussão foi importante para mostrar onde a metodologia precisa ser melhorada e quais são os pontos que ainda estão em aberto, conhecer o estado-da-arte. Porém apesar desses pontos em aberto, muitas são as abordagens que pode ser feitas com essas ondas de corpo, tanto metodologias diferentes quanto aplicações inovadoras. 

\bibliographystyle{seg.bst}
\bibliography{References.bib}
    
\end{document}
