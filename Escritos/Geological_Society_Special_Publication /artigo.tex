%\documentclass[paper,twocolumn]{geophysics}
%\documentclass[manuscript]{geophysics}
\documentclass[long]{geophysics}
% An example of defining macros

\usepackage[english]{babel}
\usepackage[T1]{fontenc}
\usepackage[utf8]{inputenc}



\usepackage{amsmath}


\newcommand{\rs}[1]{\mathstrut\mbox{\scriptsize\rm #1}}
\newcommand{\rr}[1]{\mbox{\rm #1}}
\usepackage{color}

\begin{document}

\begin{center}
\textbf{\LARGE
 Artigo 06 - On the stress system that formed the Laramide Wind River Mountains, Wyoming - \cite{brewer_stress_1980}} \\
\textit{Diogo Luiz de Oliveira Coelho}
\footnote{\textit{Universidade Federal do Rio Grande do Norte - UFRN
Centro de Ciências Exatas e da Terra - CCET
Departamento de Geofísica
Campus Universitário - Lagoa Nova
59072-970 Natal, RN}}
\end{center} 


\section{Introduction}



\section{Geological Setting}
The Parnaíba basin of NE Brazil is one of three large Phanerozoic sedimentary basins in northern South America; Parana, Parnaíba, and Amazon (Figure 1). The depth to basement image of Figure 2, derived from potential field data and constrained by wells and seismic data, gives an impression of the differing shape, size, and depth of these basins. The Parnaíba basin is located between the Amazonian craton and São Francisco craton [Almeida et al., 1981; Brito Neves and Fuck, 2013; Cordani et al., 2013]. It is underlain by relatively thick lithosphere of the order of 160–180 km [McKenzie and Priestley, 2008, personal communication,  2014], and Precambrian crust accreted and stabilized during the Neoproterozoic Brasiliano orogeny [Brito Neves et al., 1984]. Fortes [1978] postulated that the form and tectonic history of the Parnaíba basin
are related to the reactivation of preexisting basement structures. On the basis of regional geology and isotope studies of a few core samples, Brito Neves et al. [1984] proposed the existence of a distinct but unexposed basement block beneath the center western part of the basin, trapped between the Amazonian craton and the Borborema orogenic belt (Figure 1).

Góes and Feijó [1994] described the Parnaíba basin as occupying over
600,000 km 2 and comprising up to ~3.4 km of Phanerozoic sedimentary
section overlying localized rifts. Most recently, de Castro et al. [2014]
extended this perspective, arguing from airborne gravity and magnetic data, that
most of the Parnaíba basin is underlain by Eopaleozoic rifts. They concluded
that rifting was the driving mechanism for the subsidence of the Phanerozoic
basin. Two major igneous events punctuate the basin development, with
extensive extrusives in the Early Jurassic and widespread dykes and sills of Early Cretaceous age [Fodor et al., 1990].

\section{Main tectonic features}

Continental-scale shear zones (lineaments) played a major
role in the Brasiliano orogeny and in the evolution of the
Parnaíba Basin. These shear zones mark sutures associated
with continental collisions such as the Araguaia and Trans-
brasiliano lineaments (Fig. 2). The 1000 km long Araguaia
suture zone represents the final Neoproterozoic collision be-
tween the Amazonian craton, overlain by the allochthonous
Araguaia belt, and the pre-Neoproterozoic Parnaíba block
(Brito Neves and Fuck, 2014). Another important shear zone
is the Transbrasiliano Lineament. Many studies considered
the Transbrasiliano Lineament to be a continental-scale dis-
continuity characterized by strong long-wavelength magnetic
anomalies and by low S wave velocities in the mantle (e.g.,
Fairhead and Maus, 2003; Feng et al., 2004; Fuck et al.,
2008; Brito Neves and Fuck, 2014). On the NE side of the
Parnaíba basin margin, the Transbrasiliano Lineament sepa-
rates two Neoproterozoic crustal domains of the Borborema
province (Médio Coreaú and Ceará Central; Fig. 2). The
Transbrasiliano Lineament also controlled the internal rift
geometry and formed a 150 km wide rift zone in the east-
ern portion of the basin. Later reactivations of the Brasil-
iano shear zones deformed post-rift sequences, including

post-Devonian tectonic inversion (Destro et al., 1994). These
lineaments also form Precambrian lithospheric-scale bound-
aries. They were identified in a deep crustal, seismic reflec-
tion profile across the Parnaíba basin (Daly et al., 2014) and
represent the collisional sutures of the Amazonian and the
São Francisco cratons (de Castro et al., 2014).
Following the Brasiliano/Pan-African orogeny, tectonic
inversion generated elongated grabens controlled by Precam-
brian structural fabric, which is mainly marked by ductile
shear zones in the basement. The best examples of these
grabens are the Jaibaras basin and other smaller Cambrian–
Ordovician rift basins that are partially exposed at the north-
ern and eastern edges of the Parnaíba basin (Fig. 2). The
Jaibaras is the best known of these basins. It crops out at
the NE boundary of the Parnaíba basin (Fig. 2), forms a
NE-trending, 120 km long and 10 km wide graben generated
by the reactivation of the Transbrasiliano Lineament in the
Cambrian–Ordovician basin (Oliveira and Mohriak, 2003).


\section{P   receiver   function}


The   teleseismic   P   receiver   function   method   has   become   a   popular  technique   to constrain   crustal   and   upper   mantle   velocity   discontinuities   under a   seismic   station (e.g.   Langston,   1977;   Owens   et   al.,   1984;   Kind   and   Vinnik,   1988;   Ammon, 1991; Kosarev   et   al,   1999;   Yuan   et   al.,   2000).  Telesismic   body   wavefor
ms   recorded   at   a three-component seismic  station  contain  a wealth  of information  on  the earthquake source, the earth structure in the vicinity of both source and the receiver, and mantle propagation   effects.   The   resulting   receiver   function   is   obtained   by   removing   the
effects of source and mantle path.

The basic aspect of this method is that a few percent of the incident P wave energy from teleseismic events at significant and relatively sharp velocity discontinuities in the crust and upper mantle will be converted to S wave (Ps
), and arrive at the station within the P wave coda directly after the direct P wave (Fig. 3.1). Ps converted waves are   best   observed   at   epicentral   distances   between   30  and   95 degree   and   are   contained largely on the horizontal components. The amplitude, arrival time, and polarity of the
locally   generated   Ps   phases   are   sensitive   to   the   S-veloci
ty   structure   beneath   the recording station. The   data   which   satisfied   the   following   conditions   have   been   used   to   compute   P
receiver functions.
       1. Epicentral distances between 30-95 degree
       2. Magnitude larger than 5.5 (mb)
       3. clear P onset with high signal-to-noise ratio

Receiver functions are time series obtained from
teleseismic P-waveforms recorded at single seismic stations,
after deconvolving the vertical components from the
corresponding horizontal components [Langston, 1979].
The deconvolution operation effectively removes the signa-
ture of the common source time function and instrument
response from the resulting trace, leaving only the signature
of the near-receiver propagation. The deconvolved traces
can be regarded as linear combinations of peaks and troughs
representing secondary energy generated after the interac-tion of the incoming teleseismic P-wavefront with subsur-
face discontinuities. The interaction generally results into a
Ps converted phase (i.e., a P-to-S conversion upon refraction
across the discontinuity) and two multiply reverberated
phases between the discontinuity and the free surface (PpPs,a reverberation with two P-segments and one S-segment in
the raypath; PsPs + PpSs, a reverberation with one P-
segment and two S-segments in the raypath). The analysis
of the amplitudes and traveltimes of the interaction phases
provides important constraints on the seismic structure
under the station [see, e.g., Owens et al., 1984; Ammon et
al., 1990; Zandt et al., 1995; Zhu and Kanamori, 2000].

For this study we have selected teleseismic P-wave-
forms recorded by the BLSP stations shown in Figure 1,
with sources in the 30 < D < 90 epicentral distance range
and body wave magnitudes above 5.5. The stations used
three-component STS-2 sensors with flat velocity response
from 0.008 to 50 Hz. Additionally, we have also included
teleseismic PP-waveforms (i.e., P-waveforms that reflect off
the free-surface once) for epicentral distances beyond 50
and body wave magnitudes above 5.5. Adding the PP-
waveforms allowed a more complete azimuthal coverage
of seismic sources around the station and supplemented the
data set at stations with short recording periods. The station
coordinates and recording times for the BLSP stations
considered in this study are listed in Table 1.

To compute the receiver functions, the selected
waveforms were decimated to 10 s.p.s., windowed between
10 s before and 100 s after the leading arrival (either P or
PP), de-trended, tapered, and high-pass filtered above 50 s
to remove low-frequency, instrumental noise. Radial and
transverse receiver functions were then obtained from the
filtered traces by rotating the original horizontal compo-
nents around the corresponding vertical component into the
great-circle path, and applying the iterative, time domain
deconvolution procedure of Ligorrıa and Ammon [1999] to
the rotated traces, with 500 iterations. The transverse
receiver functions were not used in our subsequent analysis,
but provided a useful measure of the degree of lateral
heterogeneity and isotropy of the propagating medium.
The iterative deconvolution procedure applies a Gaussian
low-pass filter to the original waveforms to remove high-
frequency noise. We computed receiver functions at two
overlapping frequency bands corresponding to Gaussian
widths of a = 1.0 and a = 2.5 (corner frequencies of
0.5 Hz and 1.2 Hz, respectively), since they contain
complementary information on the receiver structure under
the station [see, e.g., Julià, 2007].

The percentage of recovery of the original radial
waveform was assessed from the RMS misfit between the
original radial waveform and the convolution of the radial
receiver function with the original vertical component, and
those events recovering less than 85% were rejected.
Additionally, the remaining waveforms were visually
inspected for stability and waveform coherency. Our orig-
inal selection consisted of 5977 P-waveforms and 13642
PP-waveforms, resulting in a total of 456 P- and PP-receiver
functions with a high-frequency content (a = 2.5) and 500
P- and PP-receiver functions with a low-frequency content
(a = 1.0) after our strict quality control.

Figure 4 displays sample radial and transverse re-
ceiver function averages for all the stations utilized in this
study, and a simple inspection of the waveforms reveals
important properties of the propagating medium under the
BLSP stations. First, the transverse receiver functions
generally display small amplitudes compared to the
corresponding radial waveforms. The transverse signal in
the P-wave coda is expected to be identically zero for
laterally homogeneous media, and the small transverse
amplitudes indicate the propagating medium under the
BLSP stations is laterally homogeneous and isotropic to a
good approximation. Stations APOB, CDSB, NUPB and
perhaps JATB, on the other hand, have larger transverse
signals and this must be kept in mind when analyzing data
from these stations. Second, the signature of the sedimen-
tary cover is quite apparent in all the radial waveforms. The
shift in the main peak, for instance, is due to a large Ps
phase generated at the sediment – bedrock interface that
arrives shortly after the incoming P-wave, and cannot be
resolved by the Gaussian filter [Cassidy, 1992]. Also other
apparent peaks and troughs between 1 and 3 s are also
caused by the interaction of the impinging P-wavefront with
sedimentary structure. Finally, the Ps phase generated at the
Moho is generally apparent in all the waveforms at about 5 s,
but the multiply reverberated phases in the bulk crustal
structure are generally harder to identify. The wavelengths
of the reverberated phases are shorter than those of the Ps
phase, and a gradational crust – mantle boundary could
reduce their amplitudes significantly [Owens and Zandt,
1985; Julià, 2007].


\section{ Crustal Thickness and Bulk Vp/Vs Ratio}
The first step in our analysis consisted of obtaining
estimates for the crustal thickness and bulk Vp/Vs ratio
from the Moho interaction phases in the receiver functions
with the hk-stacking technique of Zhu and Kanamori
[2000]. This procedure performs a grid-search over a
stacking surface built by summing a weighted combination
of the Ps, PpPs and PpSs + PsPs amplitudes measured along phase moveout curves. The phase moveout curves are
computed assuming a layer over a half-space model for a
range of possible crustal thicknesses and Vp/Vs ratios, and
should intercept the peak amplitudes in the receiver func-
tions for the ‘‘true’’ values. In this procedure, the P-wave
velocity for the layer and the phase weights must be
specified a priori. Our approach has been to fix the P-wave
velocity to a value of 6.5 km/s, which is a representative
average for Precambrian terranes worldwide [e.g., Christensen
and Mooney, 1995], and to give a zero weight to phases that
are not observed in the receiver functions. Confidence
bounds for the thickness and Vp/Vs estimates have been
obtained by bootstrapping the receiver function waveforms at
each station with 200 replications [Efron and Tibshirani,
1991].

Table 2 lists the hk-stacking results for the BLSP
stations in the Paraná basin, along with other relevant
parameters. Note that station RCLB, which is located above
the surface trace of the Jacutinga fault, has been split into
two subsets sampling each side of the fault. Also note that
no values are reported for stations APOB and CCUB, due to
a small data set. The crustal thicknesses range between 41
and 48 km and are generally constrained within 2 km, the
only exceptions being stations CDSB, NUPB, RCLB(E),
and RIFB which yielded confidence bounds in the 3 km
and 4 km range. Overall, these values are in excellent
agreement with the independent estimates from previous
surface-wave and receiver function studies described in
section 2.2. The bulk Vp/Vs values, however, are more
variable and less tightly constrained. Many of the estimates
range between 1.69 and 1.76 and have confidence bounds
below 0.04, but a significant number of them have
confidence bounds between 0.06 and 0.10. This range
of Vp/Vs values is compatible with a bulk felsic composi-
tion [e.g., Christensen, 1996], but the large confidence
bounds actually allow for a broader range of crustal com-
positions. Figure 5 displays the hk-stacking surfaces and
the corresponding phase moveout curves for all the BLSP
stations considered in this study. Even though all the hk-
stacking surfaces show a prominent, single-peaked max-
imum around the estimated values, the only phase
consistently observed in all the waveforms is the Ps
refracted wave. The multiples are seen less consistently
among the waveforms, especially the PpSs + PsPs phase,
and this variability is translated into large confidence
bounds during the bootstrap resampling.

One surprising result is that obtained for station
PPDB, located along the axis of the basin, which yielded
an anomalously high Vp/Vs ratio of 1.83 0.03. A
similar value of 1.85  0.05 was reported by An and
Assumpção [2006] for this same station, from the slant
stacking of receiver functions. This agreement, along with the small confidence bounds, suggest this value is
well constrained. A Vp/Vs of 1.83 is suggestive of crust
of more mafic composition [Christensen, 1996], perhaps
due to mafic underplate, but it seems inconsistent with a
relatively thin crust of 41 1 km and with the lower
Vp/Vs values at nearby stations (see Figure 6). Also, as
shown later, a layer of mafic underplate is not observed
in the joint inversion model for this station. Another
possibility is that reverberated energy trapped in the
sedimentary structure is interfering with the Ps phase
refracted at the Moho and slightly advancing the time of
the peak amplitude. Energy trapped in sediments can
reverberate for a long time and even mask the signature
of deeper discontinuities [e.g., Julià et al., 2004]. Sedi-
ments do not seem to affect the Ps phase from the Moho
at any other station, at least not so significantly, but the
influence of the sedimentary structure must be kept in
mind when interpreting the results in Table 2.

The strongest correlation between the hk-stacking
results and subsurface structure is to sediment thickness.
Figure 6 overlays the hk-stacking results with the basement
depth isolines in Figure 2a. Note that the largest estimates in
both Vp/Vs and thickness approximately cluster along the
axis of the basin, where the sediments are thickest. By
correcting the crustal thicknesses for sedimentary structure,
we can attempt a correlation of basement thickness with the
basement models discussed in section 2.2 (Figure 7). The
only correlation we observe is to the Paranapanema block,
but this correlation is counterintuitive. Mantovani et al.
[2005] defined the Paranapanema block from a gravity high
in the Bouguer anomaly map, which was interpreted as the
signature of a central cratonic nucleus under the basin
framed by Neoproterozoic crust thickened by plate inter-
actions. Our results indicate the crust within the boundaries
of the Paranapanema block is thicker, not thinner, than the
surrounding crust. As discussed in the next section, only a
few stations show evidence of mafic underplate, and these
stations are located either outside or very close to the
borders of the basement fragments postulated by Milani
and Ramos [1998].

\section{Joint Inversion}

S-wave velocity models beneath BLSP stations in the
Paraná basin have been obtained through the iterative,
linearized inversion scheme of Julià et al. [2000, 2003].
To ensure that both data sets sample similar regions of the
Earth, we identified the surface-wave tomographic cell
enclosing each station in our study and extracted the local
group velocity curve for the tomographic cell. We then
inverted the receiver functions (high- and low-frequency)
jointly with the extracted dispersion curve. The data sets
were normalized for the different number of data points and
physical units prior to inversion. The procedure includes an
influence factor that weights the contribution of each data
set to the misfit function driving the inversion. This param-
eter was set to a value of 0.5, which provided a good
compromise between fitting the receiver functions and the
dispersion velocities.

The starting model for the linearized procedure
assumed an isotropic, perfectly elastic medium with a 40-km-
thick crust and a linear S-velocity increase from 3.4 to
4.0 km/s overlying a flattened PREM. The crustal Vp/Vs
ratio was set to an a priori value of 1.73 and the crustal
densities were calculated from the P-wave velocities
through the empirical relationship of Berteussen [1977].
The starting model was parameterized as a stack of thin
layers of constant thickness (0.25 –1.5 km down to 5 km
depth, 2.5 km down to 50 km depth, 5.0 km down to 100 km
depth, and 10 km down to 400 km depth). Layers as thin as
0.25 km are well beyond the resolving power of our high-
frequency receiver functions, but are required to closely
match the a priori geotechnical values in Table 3. Inverting
for a large number of thin layers can sometimes lead to
instabilities that drive the iterative process away from
convergence. This difficulty is generally overcome through
smoothness constraints in the velocity profiles, at the
expense of losing resolution in the inverted models [e.g.,
Ammon et al., 1990]. In our inversions, we utilized a depth-
dependent smoothing that allowed the sedimentary and
shallow basement structure to be modeled in full detail by
the data and a priori constraints while preventing instabil-
ities from arising during the inversion process. In general,
6 iterations sufficed for the inversion process to converge to
a final velocity model.

\bibliographystyle{seg.bst}
\bibliography{References.bib}
    
\end{document}
